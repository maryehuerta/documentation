Discuss the state-of-the-art with respect to your product. What solutions currently exist, and in what form (academic research, enthusiast prototype, commercially available, etc)? Include references and citations as necessary \cite{Rubin2012}. If there are existing solutions, why won't they work for your customer (too expensive, not fast enough, not reliable enough, etc.). This section should occupy 1/2 - 1 full page, and should include at least 5 references to related work.

Tomahawk Player \cite{tomahawk}
\begin{itemize}
  \item Tomahawk is a free multi-source and cross-platform music player. An application that can play not only your local files, but also stream from services like Spotify, Beats, SoundCloud, Google Music, YouTube and many others.
  \item Tomahawk is basically a player for music metadata. Given the name of a song and artist, Tomahawk will find the right source, for the right user at the right time.
  \item Tomahawk is primarily a \textbf{desktop project}, while Synthify will aim to be a \textbf{web application} so that it is available on any device that has internet.
  \item Tomahawk was an \textbf{enthusiast project}, but is essentially an \textbf{abandoned} project.
\end{itemize}

Soundiiz \cite{soundiiz}
\begin{itemize}
  \item Soundiiz is a playlist converter/manager for several music streaming sites.Several streaming platforms are available as Deezer, Apple Music, SoundCloud, YouTube, Qobuz, Spotify, Napster.
  \item Soundiiz seems to only allow you to merge playlists, and not play the content in your playlists like Synthify aims to do.
  \item Soundiiz is \textbf{commercially available}, but offers a free plan.
\end{itemize}

Amplifind \cite{amplifind}
\begin{itemize}
  \item Amplifind is a music player built by music-lovers for music-lovers. Designed to be simple and fast, Amplifind gets you to the music you want to hear whenever you want to hear it.
  \item “Until now, this meant you had to visit each site separately to find and listen to the music you want. Now you can access it all in one place.”
  \item Amplifind is a mobile application available on the Apple App Store.
  \item Works with local mp3’s, Grooveshark, Spotify, and Soundcloud.
  \item Also has a 3D music visualizer built in.
  \item Amplifind was \textbf{commercially available}, but is essentially an \textbf{abandoned} project with no updates.
\end{itemize}
